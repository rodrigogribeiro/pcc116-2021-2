\documentclass[a4paper]{article}

\usepackage[portuguese]{babel}
\usepackage[utf8]{inputenc}
\usepackage{graphicx,hyperref}
\usepackage{float}
\usepackage{proof,tikz}
\usepackage{amssymb,amsthm,stmaryrd}


\usepackage{fancyhdr}
  \pagestyle{fancy}
  \fancyhf{}
  \lhead{Lógica aplicada à computação}
  \rhead{Lista de exercícios 01}
  \lfoot{Prof. Rodrigo Ribeiro}
  \rfoot{\thepage}
  \renewcommand{\footrulewidth}{0.4pt}
  \pagestyle{fancy}

\tikzset{
        ->,  % makes the edges directed
        >=stealth', % makes the arrow heads bold
        node distance=3cm,
        every state/.style={thick, fill=gray!10},
        initial text=$\,$
        }
  

\begin{document}

  \title{Lista 01 - Lógica proposicional}
  \author{Rodrigo Ribeiro}

  \maketitle


  \pagestyle{fancy}

  \section*{Avisos sobre a entrega da lista 01}

  \begin{itemize}
    \item As listas deverão ser resolvidas utilizando~\LaTeX~usando o template fornecido.
    \item O conjunto de soluções deverá ser entregue na plataforma Moodle como um único
        arquivo \textbf{.pdf}. É de responsabilidade do aluno a entrega de sua solução dentro do
        prazo estabelecido.
  \end{itemize}

  \section*{Exercícios}

  \begin{enumerate}
    \item Para cada um dos sequentes a seguir, apresente uma demonstração em 1) dedução natural; 2) normal natural deduction.
          \begin{enumerate}
            \item $\vdash (A \supset B) \supset (A \supset B \supset C) \supset (A \supset C)$.
            \item $\{\neg (A\lor B)\} \vdash \neg A \land \neg B$.
          \end{enumerate}
    \item Apresente uma dedução em cálculo de sequentes para
          \[
            \Longrightarrow (A \supset B \supset C) \supset ((A \supset B) \supset (A \supset C))
          \]

    \item Apresente deduções usando o cálculo para proof search para as seguintes tautologias.
          Lembre-se que $\neg \varphi \equiv \varphi \supset \bot$!
          \begin{enumerate}
            \item $\neg \neg (A \lor \neg A)$
            \item $\neg \neg (\neg \neg A \supset A)$
          \end{enumerate}
  \end{enumerate}

\end{document}
