\documentclass[a4paper]{article}

\usepackage[portuguese]{babel}
\usepackage[utf8]{inputenc}
\usepackage{graphicx,hyperref}
\usepackage{float}
\usepackage{listings}
\usepackage{proof,tikz}
\usepackage{amssymb,amsthm,stmaryrd}


\usepackage[edges]{forest}
\usetikzlibrary{automata, positioning, arrows}


\newtheorem{Lemma}{Lema}
\newtheorem{Theorem}{Teorema}
\theoremstyle{definition}
\newtheorem{Example}{Exemplo}
\newtheorem{Definition}{Definição}


\usepackage{fancyhdr}
  \pagestyle{fancy}
  \fancyhf{}
  \lhead{Lógica aplicada à computação}
  \rhead{Lista de exercícios 02}
  \lfoot{Prof. Rodrigo Ribeiro}
  \rfoot{\thepage}
  \renewcommand{\footrulewidth}{0.4pt}
  \pagestyle{fancy}

\tikzset{
        ->,  % makes the edges directed
        >=stealth', % makes the arrow heads bold
        node distance=3cm,
        every state/.style={thick, fill=gray!10},
        initial text=$\,$
        }
  

\begin{document}

  \title{Lista 03 - Inferência de tipos e Sistema F}
  \author{Rodrigo Ribeiro}

  \maketitle


  \pagestyle{fancy}

  \section*{Avisos sobre a entrega da lista 03}

  \begin{itemize}
    \item As listas deverão ser resolvidas utilizando~\LaTeX~usando o template fornecido.
    \item O conjunto de soluções deverá ser entregue na plataforma Moodle como um único
        arquivo \textbf{.pdf}. É de responsabilidade do aluno a entrega de sua solução dentro do
        prazo estabelecido.
  \end{itemize}

  \section*{Exercícios}

  \begin{enumerate}
    \item O sistema F é mais expressivo que o sistema de Hindley-Milner. Apresente um termo válido
          do sistema F que não é aceito pelo sistema de Hindley-Milner.
    \item Apresente representações em sistema F para os seguintes conectivos da lógica:
          \begin{enumerate}
            \item $A\lor B$
            \item $\bot$
            \item $\exists \beta. A$
          \end{enumerate}
    \item Apresente uma função, em sistema F, para somar dois números naturais em notação de Peano.
    \item Considere a tarefa de representar o produto cartesiano no sistema F.
          \begin{enumerate}
            \item Apresente um termo do sistema F para denotar o tipo $A \times B$.
            \item Apresente uma forma normal longa para denotar o par ordenado $(a,b)$.
            \item Apresente funções para obter o primeiro e segundo componente de pares.
            \item Apresente um termo do sistema F para denotar a seguinte função sobre pares:
\begin{verbatim}
swap :: (a,b) -> (b,a)
swap (x,y) = (y,x)
\end{verbatim}
          \end{enumerate}
  \end{enumerate}
\end{document}
