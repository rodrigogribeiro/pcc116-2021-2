\documentclass[a4paper]{article}

\usepackage[portuguese]{babel}
\usepackage[utf8]{inputenc}
\usepackage{graphicx,hyperref}
\usepackage{float}
\usepackage{listings}
\usepackage{proof,tikz}
\usepackage{amssymb,amsthm,stmaryrd}


\usepackage[edges]{forest}
\usetikzlibrary{automata, positioning, arrows}


\newtheorem{Lemma}{Lema}
\newtheorem{Theorem}{Teorema}
\theoremstyle{definition}
\newtheorem{Example}{Exemplo}
\newtheorem{Definition}{Definição}


\usepackage{fancyhdr}
  \pagestyle{fancy}
  \fancyhf{}
  \lhead{Lógica aplicada à computação}
  \rhead{Lista de exercícios 02}
  \lfoot{Prof. Rodrigo Ribeiro}
  \rfoot{\thepage}
  \renewcommand{\footrulewidth}{0.4pt}
  \pagestyle{fancy}

\tikzset{
        ->,  % makes the edges directed
        >=stealth', % makes the arrow heads bold
        node distance=3cm,
        every state/.style={thick, fill=gray!10},
        initial text=$\,$
        }
  

\begin{document}

  \title{Lista 02 - $\lambda$-cálculo}
  \author{Rodrigo Ribeiro}

  \maketitle


  \pagestyle{fancy}

  \section*{Avisos sobre a entrega da lista 02}

  \begin{itemize}
    \item As listas deverão ser resolvidas utilizando~\LaTeX~usando o template fornecido.
    \item O conjunto de soluções deverá ser entregue na plataforma Moodle como um único
        arquivo \textbf{.pdf}. É de responsabilidade do aluno a entrega de sua solução dentro do
        prazo estabelecido.
  \end{itemize}

  \section*{Exercícios}

  \begin{enumerate}

    \item O $\lambda$-cálculo não tipado pode ser utilizado para representar quaisquer funções computáveis.
          O objetivo deste exercício é a definição de algumas funções usando esse formalismo.
    \begin{enumerate}
      \item Apresente a definição de um $\lambda$-termo para realizar a soma de dois números naturais
            expressos usando a notação de Peano.
      \item Seja $add$ o termo definido por você no item anterior.
            Apresente a redução do termo $add\:\overline{1}\:\overline{1}$, em que $\overline{n}$ denota
            a representação de $n\in\mathbb{N}$ em notação de Peano.
      \item Apresente a definição de um $\lambda$-termo para retornar o maior dentre dois números naturais
            expressos usando a notação de Peano.
      \item Apresente uma definição de um $\lambda$-termo para calcular o número de elementos presentes em uma lista.
            Sua definição deverá considerar a representação de listas em $\lambda$-cálculo não tipado utilizada
            nas aulas.
      \item Apresente como definir árvores binárias utilizando termos do $\lambda$-cálculo não tipado.
      \item Apresente um $\lambda$-termo para calcular a altura de árvores binárias representadas utilizando a notação
            definida por você no item anterior.
    \end{enumerate}
    \item A seguinte proposição é uma tautologia da lógica proposicional:

          \[
          (D \to (A \to C) \to A) \to (A\to C) \to D \to C
          \]
          \begin{enumerate}
            \item Apresente o $\lambda$-termo correspondente a essa tautologia da lógica,
                  de acordo com a correspondência de Curry-Howard.
            \item Apresente a derivação de tipos para o termo apresentado por você no item anterior.
          \end{enumerate}
    \item Considere o seguinte termo:
          \[
          \vdash \lambda x : \tau_{1}. \lambda y : \tau_{2}.\lambda z : \tau_{3}. x\:y\:(y z) : \tau_{4}
          \]
          Faça o que se pede.
          \begin{itemize}
            \item Apresente tipos $\tau_{1},\tau_{2}, \tau_{3}$ e $\tau_{4}$ que permitam a construção de
                  uma derivação de tipos para este termo.
            \item Apresente a derivação de tipos para os tipos escolhidos por você no item anterior.
          \end{itemize}
  \end{enumerate}
\end{document}
